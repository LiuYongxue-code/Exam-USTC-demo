\documentclass[space]{ctexart} 
\usepackage{geometry,calc}
\geometry{a3paper,twocolumn,landscape,hmargin={1.5cm,2cm},vmargin={2cm,2cm},footskip=0.3cm,headsep=1cm}

\usepackage{enumitem}
\usepackage{amsmath,amssymb}%
\def\dif{\mathop{}\!\mathrm{d}}
\DeclareMathOperator{\rank}{rank}
\usepackage{bm}
\usepackage{pifont}
\usepackage{extarrows}
\usepackage{scalerel}

\usepackage{fancybox}
\fancypage{
\setlength{\fboxsep}{8pt}%
\setlength{\fboxrule}{1.5pt}%
\setlength{\shadowsize}{0pt}%
\shadowbox}{}

\usepackage{fancyhdr}
\pagestyle{fancy}
\fancyhf{}
\renewcommand{\headrulewidth}{0pt}
\setlength{\columnseprule}{0.4pt}
\setlength{\columnsep}{1.8cm}

\topskip=1.5cm

\usepackage{tikz}
\def\univ#1{\begin{center}{\zihao{-1}\CJKfontspec{华文中宋粗} #1}\end{center}}
\def\biaoti#1{\begin{center}{
    \begin{tikzpicture}[overlay]
    \node at (4.75,1.7){\zihao{1}\CJKfontspec{华文中宋粗} 中~ 国~ 科~ 学~ 技~ 术~ 大~ 学};
    \end{tikzpicture}
    \zihao{2}\CJKfontspec{华文中宋粗} #1}\end{center}}
\def\DMKM{
\begin{center}
{\zihao{-4}\heiti
姓名: \underline{\hspace{4cm}}\hspace{2cm}学号: \underline{\hspace{4cm}}
}
\end{center}}
\newcommand\DT[2]
{
\begin{center}
    {
        \hspace{2cm}
    {\zihao{-4}\heiti 考试时间:} {\zihao{-4} #1} \hfill {\zihao{-4}\heiti 主讲教师:} {\zihao{-4} #2}
        \hspace{2cm}
    }
\end{center}
}
\def\tishi#1{\begin{center}{\zihao{-4}\heiti #1}\end{center}}

\usepackage{lastpage}
\newcommand{\lastpageref}{\pageref{LastPage}}
\usepackage{hyperref}
\newcounter{foo}
\newcounter{fox}
\addtocounter{foo}{1}
\fancyfoot[CE,CO]{\hspace*{17.75cm}第\refstepcounter{fox}\thefoo\refstepcounter{foo}页 \quad 共~\lastpageref~页\hspace*{9cm}
\textsl{考试结束后请将本试题与答题纸(卡)一并交回}}
\renewcommand{\headrulewidth}{0pt}

\ctexset{section = {name = {,、\hspace*{-5mm}},number = \chinese{section},format = {\vspace{4mm}\zihao{4}\heiti\raggedright}}}
\ctexset{subsection = {name = {\hspace{5mm},\hspace*{-1mm}},number = (\chinese{subsection}),format = {\zihao{4}\heiti\raggedright}}}
\usepackage{titlesec}
\titlespacing*{\section}{0pt}{9pt}{4pt}

%%--------------------------设置数学字体为新罗马字体-------------------------
\RequirePackage[T1]{fontenc}
\RequirePackage{mathptmx}
\newfontfamily\TNRBI{Times New Roman Bold Italic} %新罗马字体英文的粗斜体
\newcommand{\bfit}[1]{\text{\TNRBI #1}} %amsmath 
%%-------------------------------------------------------------------------

%选择题
\newcommand{\fourch}[4]{\\\begin{tabular}{*{4}{@{}p{3.5cm}}}(A)~#1 & (B)~#2 & (C)~#3 & (D)~#4\end{tabular}} % 四行
\newcommand{\twoch}[4]{\\\begin{tabular}{*{2}{@{}p{7cm}}}(A)~#1 & (B)~#2\end{tabular}\\\begin{tabular}{*{2}{@{}p{7cm}}}(C)~#3 &
		(D)~#4\end{tabular}}  %两行
\newcommand{\onech}[4]{\\(A)~#1 \\ (B)~#2 \\ (C)~#3 \\ (D)~#4}  % 一行

\usepackage{makecell}
\usepackage{setspace} 


\def\tiankongdaan#1{\makebox[3em][c]{#1}}
\newcommand{\blank}[1]{\underline{\tiankongdaan{\hspace{4cm}}}}
\DeclareOption{ans}{\renewcommand{\blank}[1]{\,\underline{#1}\,
}}
\ProcessOptions


\begin{document}

\setcounter{page}{9} %这里设置从第九页开始计数
% ***************************************
\univ{~}
\biaoti{2023--2024 秋季实分析试卷}
\DT{\today 8:30--10:30}{赵老师、郭老师}
\DMKM
\tishi{注意: 所有题目的解答要有详细过程, 其中使用的定理或命题需要注明.}

\begin{center}
\begin{tabular}{|m{0.03\textwidth}|*{8}{m{0.035\textwidth}|}p{0.04\textwidth}|}
	\hline
\centering  题~号 & \centering 一 & \centering 二 & \centering 三 & \centering 四& \centering 五 & \centering 六 & \centering 七 % & \centering 八 & \centering 九 & \centering 十
& \centering 总~分 & \makecell{阅卷\\教师} \rule{0pt}{3mm} \\
	\hline
	\centering 分~数 &  &  &  &  &  &  &  &  &  %&  &
	\rule{0pt}{8mm} \\\hline
				% \centering 计 &  &  &  &  &  &  &  &  &  &  & \\
				% \centering 分 &  &  &  &  &  &  &  &  &  &  & \\
				% \centering 人 &  &  &  &  &  &  &  &  &  &  & \\  \hline
\end{tabular}
\end{center}
\setlength{\parskip}{1.5ex} % 大标题与小标体之间固定间距
\section*{\hspace{5cm} 一、选择题~(每题~10 分,共~10 分)}
\vspace{-1.3cm} \hspace{0.1cm} 
		\begin{tabular}{|p{0.05\textwidth}|p{0.05\textwidth}|}
			\hline
			% after \\: \hline or \cline{col1-col2} \cline{col3-col4} ...
			\centering 阅卷人& \\
			\hline
			\centering 得~~分 &  \\
			\hline
		\end{tabular}



\setcounter{enumii}{0}
\begin{enumerate}[itemsep=1.2em,topsep=0pt,left=2em] % itemsep 参数是两个 item 内容之间的间距
\item 设有平面区域~$D=\{(x,y)\mid -a\leqslant x\leqslant a, x\leqslant y\leqslant a\}$ , ~$D_1=\{(x,y)\mid 0\leqslant x\leqslant a, x\leqslant y\leqslant a\}$ , 则~$\displaystyle{\iint\limits_{D}(xy+\cos x\sin y)\dif x\dif y}=$ ~(~~~~~~)
 \twoch{~$0$.}{~$4\displaystyle{\iint\limits_{D_1}(xy+\cos x\sin y)\dif x\dif y}$.}{~$2\displaystyle{\iint\limits_{D_1}xy\dif x\dif y}$.}{~$2\displaystyle{\iint\limits_{D_1}\cos x\sin y\dif x\dif y}$.}

%\end{enumerate}


\section*{\hspace{4cm} 二、填空题~(每题~10 分,共~20 分)}
\vspace{-1.3cm} \hspace{-1cm} 
		\begin{tabular}{|p{0.05\textwidth}|p{0.05\textwidth}|}
			\hline
			% after \\: \hline or \cline{col1-col2} \cline{col3-col4} ...
			\centering 阅卷人& \\
			\hline
			\centering 得~~分 &  \\
			\hline
		\end{tabular}


%\begin{enumerate}[itemsep=1.2em,topsep=0pt,left=2em]
%\setcounter{enumii}{1}

  \item 设~$z = u^2\ln v$ , 而~$u= \dfrac{x}{y}, v = x-y$ , 则~$\dfrac{\partial z}{\partial x}= $~\blank{~~$\dfrac{2x}{y^2}\ln(x-y)+\dfrac{x^2}{y^2(x-y)}$~~}.
  
  \item 把二次积分~$\displaystyle{\int_0^1 \dif x \int_0^{\sqrt{1-x^2}} f(x,y) \dif y}$ 化为极坐标形式的二次积分为\blank{~~~$\displaystyle{\int_0^{\pi/2} \dif \theta \int_0^1 f(\rho \cos\theta, \rho\sin\theta)\rho \dif \rho}$~~~~}.

\section*{\hspace{4cm} 三、解答题~(每题~30 分,共~120 分)}
\vspace{-1.3cm} \hspace{-1cm} 
		\begin{tabular}{|p{0.05\textwidth}|p{0.05\textwidth}|}
			\hline
			% after \\: \hline or \cline{col1-col2} \cline{col3-col4} ...
			\centering 阅卷人& \\
			\hline
			\centering 得~~分 &  \\
			\hline
		\end{tabular}
  
\item (30 分) 设$f$是$\mathbb{R}^\mathrm{d}$上不恒为零的可积函数, 证明存在常数$c>0$使得对于所有的$|x|\geq 1$,
\begin{align*}
f^*\geq \dfrac{c}{|x|^{\rm d}}
\end{align*}
其中
\begin{align*}
f*(x)=\sup\limits_{x\in B}\dfrac{1}{m(B)}\int_{B}|f(x)|\dif y,~x\in \mathbb{R}^\mathrm{d}.
\end{align*}


\item (30分) 设 $ f_{n} $ 是区间$[0,1] $ 上的一列可测函数, 满足
\begin{align*}
\lim_{n \to \infty} f_{n}(x)=0, ~\text{a.e.}~ x \in [0,1]
\end{align*}
且
\begin{align*}
\sup\limits_{n}\left\|f_{n}\right\|_{L^{2}([0,1 \mid)} \leq 1
\end{align*}
证明:
\begin{align*}
\lim\limits_{n \to \infty}\left\|f_{n}\right\|_{L^{1}([0,1])}=0.
\end{align*}

\item (30分) 令$ m $ 表示 $ \mathbb{R}$ 上的Lebesgue测度,  $A \subset \mathbb{R} $ 是Lebesgue 可测集. 假设对于所有的实数  $a<b$,
    \begin{align*}
    m(A \cap[a, b])<\dfrac{b-a}{2}
    \end{align*}
    证明: $ m(A)=0 $.

\item (30分) 若 $ f $ 在$\mathbb{R}$上绝对连续, 且 $ f \in L^{1}(\mathbb{R}) $.  如果
\begin{align*}
\lim\limits_{t \to  0} \int_{\mathbb{R}}\left|\frac{f(x+t)-f(x)}{t}\right| \dif x=0
\end{align*}
证明: $ f \equiv 0 $.

\end{enumerate}

\clearpage	

\end{document}

